\newpage
\anonsection{Введение}

В настоящее время большую популярность имеют поисковики, в которых для удобства
пользователя необходимо исправлять ошибки. Подобные ошибки в словах могут возникнуть
также в базах данных, при распозновании отсканированного текста или речи, или
при обычном вводе текста. Подобная проблема решается с помощью редакционного
расстояния \cite{habr}.
Самыми популярными алгоритмами поиска редакционного расстояния являются
расстояние Левенштейна и Дамерау-Левенштейна, которые активно применяются:

\begin{itemize}
    \item для исправления ошибок в слове (в поисковых системах, базах
        данных, при вводе текста, при автоматическом распознавании
        отсканированного текста или речи) \cite{habr};

    \item для сравнения текстовых файлов утилитой diff и ей подобными;

    \item в биоинформатике для сравнения генов, хромосом и белков \cite{bio}.
\end{itemize}

В данной лабораторной работе ставятся следующие задачи:

\begin{enumerate}
    \item изучение алгоритмов Левенштейна и Дамерау-Левенштейна нахождения
        расстояния между строками;
    \item получение практических навыков реализации указанных алгоритмов:
        двух алгоритмов в матричной версии и оного из алгоритмов в рекурсивной
        версии;
    \item сравнительный анализ линейной и рекурсивной реализации выбранного
        алгоритма определения расстояния между строками по затрачиваемым
        ресурсам (времени и памяти);
    \item экспериментальное подтверждение различий во временой эффективности
        рекурсивной и нерекурсивной реализации выбранного алгоритма
        определения расстояния межлу строками при помощи разработанного
        программного обеспечения на материале замеров процессорного времен
        выполнения реализации на варьирующихся длинах строк.
\end{enumerate}
