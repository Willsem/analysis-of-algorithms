\newpage
\anonsection{Заключение}

В ходе данной работы было выполнено сравнение трех алгоритмов нахождения минимального редакционного расстояния для двух строк: расстояния Левенштейна матричным методом, расстояния Дамерау-Левенштейна матричным и рекурсивным методами. Данные алгоритмы широко используются для поисковиков, в которых необходимо выявлять оошибки при наборе текста, для сравнения белков и генов в биоинформатике, а также для утилит diff.

Было проведено сравнение времени всех алгоритмов для случая, когда две строки имееют одинаковую длину и для случая, когда одна строка небольшая, а вторая варьрируется. В ходе этого исследования были сделаны следующие выводы:

\begin{enumerate}
    \item[1)] рекурсивная реализация сильно проигрывает матричной реализации, и ее невыгодно использовать ни в одном случае;
    \item[2)] матричный алгоритм нахождения расстояния Дамерау-Левенштейна работает на 25\% медленнее, чем Левенштейна за счет рассчета дополнительного действия смены соседних букв, что дает лучший результат.
\end{enumerate}
