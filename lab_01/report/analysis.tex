\newpage
\section{Аналитическая часть}

Редакционное расстояние активно применяется в программировании и биоиформатике.
Самые популярные алгоритмы поиска редакционного расстояния это алгоритм
Левенштейна и Дамерау-Левенштейна \cite{habr}.

\subsection{Описание задачи}

\textbf{Редакционное расстояние (расстояние Левенштейна)}
между двух строк - это
минимальное количество операций вставки одного символа, удаления одного
символа или замены одного символа на другой необходимых для превращения
одной строки в другую. \cite{itmo}

Для двух строк $S_1$ и $S_2$ с длинами $n$ и $m$ соотвественно можно
построить матрицу $D(n+1, m+1)$, в которой каждый элемент вычисляется по формуле (1)
\cite{itmo}

\begin{equation}
D(i,j) =
\begin{cases}
    0, \text{ если } i = 0, j = 0 \\
    i, \text{ если } j = 0 \\
    j, \text{ если } i = 0 \\
    \min
    \left(
        \begin{matrix}
            D(i, j - 1) + 1 \\
            D(i - 1, j) + 1 \\
            D(i - 1, j - 1) +
            \begin{cases}
                0, \text{ если } S_1[i] \ne S_2[j] \\
                1, \text{ иначе}
            \end{cases}
        \end{matrix}
    \right)
    , \text{ иначе}
\end{cases}
\end{equation}

После нахождения каждого элемента матрицы, редакционное расстояние между
$S_1$ и $S_2$ будет в нижнем правом элементе матрицы.

\textbf{Расстояние Дамерау-Левенштейна} - это редакционное расстояние,
к которому добавляется еще одно действие - транспозиция (смена местами
двух соседних символов). Необходимо это потому, что около 80\% ошибок
при наборе текста человеком является транспозиция, как показал Дамерау. \cite{bmstu}

По аналогии с расстоянием Левенштейна, здесь тоже используется матрица
$D(n+1, m+1)$ для строк $S_1$ и $S_2$ с длинами $n$ и $m$ уже по формуле (2)
\cite{bmstu}

\begin{equation}
D(i,j) =
\begin{cases}
    0, \text{ если } i = 0, j = 0 \\
    i, \text{ если } j = 0 \\
    j, \text{ если } i = 0 \\

    \min
    \left(
        \begin{matrix}
            D(i, j - 1) + 1 \\
            D(i - 1, j) + 1 \\
            D(i - 1, j - 1) +
            \begin{cases}
                1, \text{ если } S_1[i] \ne S_2[j] \\
                0, \text{ иначе}
            \end{cases} \\
            D(i - 2, j - 2) + 1
        \end{matrix}
    \right) \\
    , \text{ если } i > 1, j > 1, S_1[i] = S_2[j - 1], S_2[j] = S_1[i - 1] \\

    \min
    \left(
    \begin{matrix}
        D(i, j - 1) + 1 \\
        D(i - 1, j) + 1 \\
        D(i - 1, j - 1) +
        \begin{cases}
            1, \text{ если } S_1[i] \ne S_2[j] \\
            0, \text{ иначе}
        \end{cases}
    \end{matrix}
    \right)
    , \text{ иначе}
\end{cases}
\end{equation}

Впервые эту задачу поставил советский математик Владимир Левенштейн при
изучении последовательностей 0-1 \cite{binlev}.

После него Фредерик Дамерау доказал, что пользователи совершают очень много
ошибок транспозиции при наборе текста, и доработал алогоритм Левенштейна.
Поэтому расстояние Дамерау-Левенштейна дает лучше результаты, чем
обычное редакционное расстояние.

\subsection{Выводы}

Редакционное расстояние широко применяется в областях программирования и
биоинформатики. Для алгоритмов Левенштейна и Дамерау-Левенштейна
используются операции вставки символа, удаления символа, замены символа и
транспозиции двух соседних символов. В данной работе будет проведено
исследование и сравнение этих алгоритмов.
