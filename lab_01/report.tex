\documentclass[a4paper,12pt]{article}
\usepackage[T2A]{fontenc}
\usepackage[utf8]{inputenc}
\usepackage[english,russian]{babel}
\usepackage{listings}

\usepackage{amsmath}
\usepackage{MnSymbol}
\usepackage{wasysym}
\usepackage{indentfirst}

\usepackage{pgfplots}
\pgfplotsset{compat=1.9}

\usepackage{geometry}
\geometry{left=2cm}
\geometry{right=1.5cm}
\geometry{top=1cm}
\geometry{bottom=2cm}

\lstset{
    language=C++,
    numbers=left,
    frame=single,
    texcl=true
}

\begin{document}

    \begin{titlepage}

        \begin{center}
            \large
            Государственное образовательное учреждение высшего профессионального образования\\
            “Московский государственный технический университет имени Н.Э.Баумана”
            \vspace{3cm}
            
            \textsc{Дисциплина: Анализ алгоритмов}
            \vspace{0.5cm}
                
            \textsc{Лабораторная работа №1}
            \vspace{3cm}
            
            {\LARGE РЕДАКЦИОННОЕ РАССТОЯНИЕ}
            \vspace{3cm}
            
            Студент группы ИУ7-53,\\   
            Степанов Александр
            \vfill
            
            2019 г.
            
        \end{center}

    \end{titlepage}

    \tableofcontents
    \newpage

    \section{Введение}
    \subsection{Постановка задачи}

    Изучить, реализовать и сравнить три версии алгоритма для поиска минимального редакционного расстояния:

    \begin{enumerate}
        \item Матричный алгоритм Левенштейна с тремя операциями (вставка, удаление, замена)
        \item Матричный алгоритм Дамерау-Левенштейна с четыремя операциямя (вставка, удаление, замена, обмен двух соседних букв)
        \item Рекурсивный алгоритм Дамерау-Левенштейна с четыремя операциями
    \end{enumerate}

    \newpage
    \section{Аналитическая часть}
    \subsection{Расстояние Левевнштейна и Дамерау-Левенштейна}

    Редакционное расстояние или расстояние Левенштейна между двумя строками - это минимальное количество операций вставки, удаления и замены одного символа, необходимых для превращения одной строки в другую. В случае с расстоянием Дамерау-Левенштейна добавляется еще одно действие - обмен двух соседних букв.

    \newpage
    \section{Конструкторская часть}
    \subsection{Описание алгоритмов}

    Матричный алгоритм расстояния Левенштейна заключается в том, что для двух строк $S_1[1..len_1]$ и $S_2[1..len_2]$ создается матрица $D$ размерностью $len_1 + 1$ на $len_2 + 1$ в которой каждый элемент соответствует формуле:

    $$
    D(i,j) = \min 
    \begin{cases}
        D(i, j-1) + 1 \\
        D(i-1, j) + 1 \\
        D(i-1, j-1) + C_\text{замены}
    \end{cases}, \text{ где } 
    $$

    $$
    C_\text{замены} = 
    \begin{cases}
        1, \text{ если } S_1[i] \ne S_2[j] \\
        0, \text{ иначе}
    \end{cases}
    $$

    После нахождения каждого элемента матрицы, редакционное расстояние между $S_1$ и $S_2$ будет в нижнем правом элементе матрицы.

    Аналогично работает матричный алгоритм Дамерау-Левенштейн, только добавляется новое слагаемое при нахождении минимума

    $$
    D(i,j) = \min 
    \begin{cases}
        D(i, j-1) + 1 \\
        D(i-1, j) + 1 \\
        D(i-1, j-1) + C_\text{замены} \\
        D(i-2, j-2) + C_\text{транспозиции}
    \end{cases}, \text{ где }
    $$

    $$
    C_\text{транспозиции} = 
    \begin{cases}
        1, \text{ если } S_1[i] = S_2[j - 1] \text{ и } S_1[i-1] = S_2[j] \\
        \infty, \text{ иначе}
    \end{cases}
    $$

    Рекурсивный алгоритм Дамерау-Левенштейна это первоначальная версия матричного алгоритма, в ней рекурсивно вызываются 4 случая действий, значения которых в матрице брались из соседних клеток

    $$
    D(S_1[1..i], S_2[1..j]) = \min 
    \begin{cases}
        D(S_1[1..i], S_2[1..j-1]) + 1 \\
        D(S_1[1..i-1], S_2[1..j]) + 1 \\
        D(S_1[1..i-1], S_2[1..j-1]) + C_\text{замены} \\
        D(S_1[1..i-2], S_2[1..j-2]) + C_\text{транспозиции}
    \end{cases}
    $$

    \newpage
    \section{Технологическая часть}
    \subsection{Листинг кода}

    \begin{lstlisting}[caption=Левенштейна матричный]
std::vector< std::vector< int > > Lmatrix::find(std::string s1,
                                                std::string s2)
{
    std::vector< std::vector< int > > matrix;
    for (int i = 0; i < s1.size() + 1; ++i) {
        matrix.push_back(std::vector< int >(s2.size() + 1));
    }

    for (int i = 0; i < matrix.size(); ++i) {
        matrix[i][0] = i;
    }

    for (int j = 0; j < matrix[0].size(); ++j) {
        matrix[0][j] = j;
    }

    for (int i = 1; i < matrix.size(); ++i) {
        for (int j = 1; j < matrix[0].size(); ++j) {
            int u = matrix[i - 1][j] + 1;
            int l = matrix[i][j - 1] + 1;
            int lu = matrix[i - 1][j - 1];
            if (s1[i - 1] != s2[j - 1]) lu++;

            matrix[i][j] = std::min(std::min(u, l), lu);
        }
    }

    return matrix;
}
    \end{lstlisting}

    \begin{lstlisting}[caption=Дамерау-Левенштейна матричный]
std::vector< std::vector< int > > DLmatrix::find(std::string s1,
                                                 std::string s2)
{
    std::vector< std::vector< int > > matrix;
    for (int i = 0; i < s1.size() + 1; ++i) {
        matrix.push_back(std::vector< int >(s2.size() + 1));
    }

    for (int i = 0; i < matrix.size(); ++i) {
        matrix[i][0] = i;
    }

    for (int j = 0; j < matrix[0].size(); ++j) {
        matrix[0][j] = j;
    }

    int inf = std::max(s1.size(), s2.size()) + 1;

    for (int i = 1; i < matrix.size(); ++i) {
        for (int j = 1; j < matrix[0].size(); ++j) {
            int u = matrix[i - 1][j] + 1;
            int l = matrix[i][j - 1] + 1;
            int lu = matrix[i - 1][j - 1];
            int lluu = inf;
            if (i > 1 && j > 1) lluu = matrix[i - 2][j - 2];

            if (s1[i - 1] != s2[j - 1]) lu++;
            if (lluu != -1 && 
                s1[i - 1] == s2[j - 2] && 
                s1[i - 2] == s2[j - 1]) 
                lluu++;
            else lluu = inf;

            matrix[i][j] = std::min(std::min(u, l), 
                                    std::min(lu, lluu));
        }
    }

    return matrix;
}
    \end{lstlisting}

    \begin{lstlisting}[caption=Дамерау-Левенштейна рекурсивный]
#include "DLrecursive.h"

int DLrecursive::find(std::string s1, std::string s2)
{
    if (s1.size() == 0) return s2.size();
    if (s2.size() == 0) return s1.size();

    return std::min(
            std::min(
                find(s1.substr(0, s1.size() - 1), s2) + 1,
                find(s1, s2.substr(0, s2.size() - 1)) + 1
            ),
            std::min(
                find(s1.substr(0, s1.size() - 1), 
                     s2.substr(0, s2.size() - 1)) + 
                (s1[s1.size() - 1] == s2[s2.size() - 1] ? 0 : 1),

                (s1.size() > 1 && 
                s2.size() > 1 && 
                s1[s1.size() - 2] == s2[s2.size() - 1] && 
                s1[s1.size() - 1] == s2[s2.size() - 2] ?
                find(s1.substr(0, s1.size() - 2), 
                     s2.substr(0, s2.size() - 2)) : 
                    int(std::max(s1.size(), s2.size()))) + 1
            )
        );
}

    \end{lstlisting}

    \subsection{Заготовленные тесты}

    Для тестирования программы были заготовлены следующие тесты

    \hfill

    \textbf{Для расстояния Левенштейна}

    \begin{tabular}{|c|c|c|}
        \hline
        $s_1$ & $s_2$ & Ожидаемый результат \\
        \hline
        word & word & 0 \\
        \hline
        word & another & 6 \\
        \hline
        ab & ba & 2 \\
        \hline
        qwerty & qwetry & 2 \\
        \hline
        abcdef & badcfe & 4 \\
        \hline
        werylongword & sh & 12 \\
        \hline
        sh & werylongword & 12 \\
        \hline
        wednesday & weekend & 5 \\
        \hline
        memory & mem & 3 \\
        \hline
        feature & erutaef & 6 \\
        \hline
    \end{tabular}

    \hfill

    \hfill

    \textbf{Для расстояния Дамерау-Левенштейна}

    \begin{tabular}{|c|c|c|}
        \hline
        $s_1$ & $s_2$ & Ожидаемый результат \\
        \hline
        word & word & 0 \\
        \hline
        word & another & 6 \\
        \hline
        ab & ba & 1 \\
        \hline
        qwerty & qwetry & 1 \\
        \hline
        abcdef & badcfe & 3 \\
        \hline
        werylongword & sh & 12 \\
        \hline
        sh & werylongword & 12 \\
        \hline
        wednesday & weekend & 5 \\
        \hline
        memory & mem & 3 \\
        \hline
        feature & erutaef & 5 \\
        \hline
    \end{tabular}

    \newpage
    \section{Экспереминтальная часть}
    \subsection{Результаты тестов}

    \textbf{Для расстояния Левенштейна матрицей}

    \begin{tabular}{|c|c|c|}
        \hline
        $s_1$ & $s_2$ & Результат \\
        \hline
        word & word & 0 \\
        \hline
        word & another & 6 \\
        \hline
        ab & ba & 2 \\
        \hline
        qwerty & qwetry & 2 \\
        \hline
        abcdef & badcfe & 4 \\
        \hline
        werylongword & sh & 12 \\
        \hline
        sh & werylongword & 12 \\
        \hline
        wednesday & weekend & 5 \\
        \hline
        memory & mem & 3 \\
        \hline
        feature & erutaef & 6 \\
        \hline
    \end{tabular}

    \hfill

    \hfill

    \textbf{Для расстояния Дамерау-Левенштейна матрицей}

    \begin{tabular}{|c|c|c|}
        \hline
        $s_1$ & $s_2$ & Результат \\
        \hline
        word & word & 0 \\
        \hline
        word & another & 6 \\
        \hline
        ab & ba & 1 \\
        \hline
        qwerty & qwetry & 1 \\
        \hline
        abcdef & badcfe & 3 \\
        \hline
        werylongword & sh & 12 \\
        \hline
        sh & werylongword & 12 \\
        \hline
        wednesday & weekend & 5 \\
        \hline
        memory & mem & 3 \\
        \hline
        feature & erutaef & 5 \\
        \hline
    \end{tabular}

    \hfill

    \hfill

    \textbf{Для расстояния Дамерау-Левенштейна рекурсией}

    \begin{tabular}{|c|c|c|}
        \hline
        $s_1$ & $s_2$ & Результат \\
        \hline
        word & word & 0 \\
        \hline
        word & another & 6 \\
        \hline
        ab & ba & 1 \\
        \hline
        qwerty & qwetry & 1 \\
        \hline
        abcdef & badcfe & 3 \\
        \hline
        werylongword & sh & 12 \\
        \hline
        sh & werylongword & 12 \\
        \hline
        wednesday & weekend & 5 \\
        \hline
        memory & mem & 3 \\
        \hline
        feature & erutaef & 5 \\
        \hline
    \end{tabular}



    \subsection{Замеры времени}

    \begin{tikzpicture}
        \begin{axis}[
            title = Две строки одинаковой длины,
            legend pos = north west,
            xlabel=len, 
            ylabel=ms,
            grid = major,
            width = 0.8\paperwidth,
            height = 0.38\paperheight,
            line width = 1
        ]
            \legend{
                Левенштейна матричный,
                Дамерау-Левенштейна матричный
            };
            \addplot[orange] coordinates {
                (1, 77) (11, 138) (21, 137) (31, 136) (41, 244) (51, 342) (61, 454) (71, 668) (81, 914) (91, 988) (101, 1181) (111, 1390) (121, 1843) (131, 2789) (141, 2821) (151, 2790) (161, 3023) (171, 3447) (181, 3742) (191, 4115) (201, 4715) (211, 4986) (221, 5906) (231, 7440) (241, 6379) (251, 7211) (261, 8436) (271, 9889) (281, 9596) (291, 9705) (301, 10259) (311, 10855) (321, 11487) (331, 12099) (341, 13012) (351, 13477) (361, 14214) (371, 19928) (381, 18360) (391, 18716) (401, 20560) (411, 19499) (421, 21418) (431, 21795) (441, 20773) (451, 21482) (461, 22131) (471, 23020) (481, 23973) (491, 24995) (501, 25870) (511, 26741) (521, 36649) (531, 36592) (541, 33299) (551, 37829) (561, 35728) (571, 36549) (581, 37639) (591, 38869) (601, 40098) (611, 41196) (621, 42438) (631, 43720) (641, 45199) (651, 47835) (661, 47733) (671, 52550) (681, 55235) (691, 53685) (701, 53348) (711, 54259) (721, 55763) (731, 57242) (741, 60176) (751, 60056) (761, 61472) (771, 63499) (781, 64570) (791, 66016) (801, 68015) (811, 69153) (821, 72821) (831, 74555) (841, 82996) (851, 79267) (861, 77447) (871, 79069) (881, 83347) (891, 82515) (901, 84359) (911, 87023) (921, 87854) (931, 91765) (941, 97067) (951, 93107) (961, 95261) (971, 96853) (981, 98622) (991, 103017)
            };

            \addplot[blue] coordinates {
                (1, 23) (11, 66) (21, 189) (31, 406) (41, 956) (51, 766) (61, 640) (71, 733) (81, 814) (91, 1095) (101, 1191) (111, 1399) (121, 1624) (131, 2263) (141, 2510) (151, 2704) (161, 3069) (171, 3380) (181, 3794) (191, 4064) (201, 4651) (211, 6175) (221, 7164) (231, 6907) (241, 6386) (251, 6655) (261, 9905) (271, 9951) (281, 10018) (291, 10662) (301, 11906) (311, 12114) (321, 12450) (331, 14286) (341, 13455) (351, 15692) (361, 14523) (371, 14969) (381, 15576) (391, 16455) (401, 17118) (411, 18057) (421, 18702) (431, 19497) (441, 20640) (451, 21278) (461, 22129) (471, 24033) (481, 24687) (491, 25239) (501, 25968) (511, 30626) (521, 31819) (531, 32102) (541, 33958) (551, 34345) (561, 35284) (571, 37315) (581, 38017) (591, 39941) (601, 42890) (611, 41521) (621, 44208) (631, 43884) (641, 47874) (651, 47748) (661, 49637) (671, 48857) (681, 50262) (691, 51516) (701, 53014) (711, 54415) (721, 58208) (731, 57416) (741, 58538) (751, 60204) (761, 61503) (771, 63808) (781, 64647) (791, 66291) (801, 70742) (811, 69107) (821, 76948) (831, 79601) (841, 74977) (851, 76988) (861, 82593) (871, 80074) (881, 81188) (891, 84003) (901, 86008) (911, 90958) (921, 101962) (931, 90376) (941, 92574) (951, 93926) (961, 96161) (971, 102527) (981, 99979) (991, 100733)
            };
        \end{axis}
    \end{tikzpicture}

    \begin{tikzpicture}
        \begin{axis}[
            title = Две строки одинаковой длины,
            legend pos = north west,
            xlabel=len, 
            ylabel=ms,
            grid = major,
            width = 0.8\paperwidth,
            height = 0.38\paperheight,
            line width = 1
        ]
            \legend{
                Дамерау-Левенштейна матричный,
                Дамерау-Левенштейна рекурсивный
            };
            \addplot[blue] coordinates {
                (1, 11) (2, 14) (3, 16) (4, 24) (5, 26) (6, 29) (7, 34) (8, 46) (9, 46)
                            };
            \addplot[red] coordinates {
                (1, 3) (2, 10) (3, 22) (4, 183) (5, 338) (6, 2218) (7, 9417) (9, 56574)

            };
        \end{axis}
    \end{tikzpicture}

    \begin{tikzpicture}
        \begin{axis}[
            title = Одна строка длины 10 вторая - переменной длины,
            legend pos = north west,
            xlabel=len, 
            ylabel=ms,
            grid = major,
            width = 0.8\paperwidth,
            height = 0.38\paperheight,
            line width = 1
        ]
            \legend{
                Левенштейна матричный,
                Дамерау-Левенштейна матричный
            };
            \addplot[orange] coordinates {
                (1, 20) (11, 30) (21, 56) (31, 69) (41, 103) (51, 123) (61, 131) (71, 183) (81, 194) (91, 212) (101, 226) (111, 244) (121, 257) (131, 337) (141, 356) (151, 477) (161, 385) (171, 400) (181, 410) (191, 425) (201, 441) (211, 455) (221, 472) (231, 515) (241, 524) (251, 624) (261, 689) (271, 692) (281, 722) (291, 724) (301, 752) (311, 752) (321, 808) (331, 881) (341, 788) (351, 1004) (361, 1003) (371, 888) (381, 849) (391, 934) (401, 1283) (411, 1159) (421, 984) (431, 936) (441, 1003) (451, 995) (461, 1010) (471, 986) (481, 1034) (491, 1018) (501, 1027) (511, 1056) (521, 1341) (531, 1323) (541, 1337) (551, 1373) (561, 1365) (571, 1381) (581, 1402) (591, 1409) (601, 1429) (611, 1440) (621, 1457) (631, 1472) (641, 1491) (651, 1514) (661, 2770) (671, 4233) (681, 1561) (691, 1573) (701, 1598) (711, 1598) (721, 1642) (731, 1629) (741, 1642) (751, 1663) (761, 1671) (771, 1696) (781, 1701) (791, 1739) (801, 1767) (811, 1782) (821, 2132) (831, 1787) (841, 1845) (851, 1828) (861, 1825) (871, 1848) (881, 1869) (891, 1901) (901, 1890) (911, 1901) (921, 1916) (931, 1939) (941, 1950) (951, 1965) (961, 1979) (971, 2021) (981, 2018) (991, 2044)
            };

            \addplot[blue] coordinates {
                (1, 26) (11, 43) (21, 73) (31, 97) (41, 139) (51, 164) (61, 187) (71, 252) (81, 270) (91, 300) (101, 298) (111, 321) (121, 342) (131, 434) (141, 593) (151, 667) (161, 1184) (171, 2057) (181, 2203) (191, 577) (201, 1330) (211, 2097) (221, 869) (231, 1295) (241, 1107) (251, 1589) (261, 1287) (271, 1119) (281, 1044) (291, 910) (301, 938) (311, 1033) (321, 984) (331, 1063) (341, 1065) (351, 1044) (361, 1063) (371, 1084) (381, 1134) (391, 1136) (401, 1152) (411, 1228) (421, 1221) (431, 1226) (441, 1240) (451, 1263) (461, 1284) (471, 1428) (481, 1335) (491, 1352) (501, 1375) (511, 1397) (521, 1682) (531, 1685) (541, 1712) (551, 1732) (561, 1752) (571, 1817) (581, 1808) (591, 1824) (601, 1843) (611, 1867) (621, 1893) (631, 1911) (641, 1936) (651, 1971) (661, 2002) (671, 2001) (681, 2329) (691, 2188) (701, 2078) (711, 2089) (721, 2127) (731, 2140) (741, 2162) (751, 2182) (761, 2202) (771, 3526) (781, 4639) (791, 2475) (801, 2365) (811, 2334) (821, 2436) (831, 2577) (841, 2575) (851, 2406) (861, 2430) (871, 2510) (881, 2472) (891, 2509) (901, 3305) (911, 2646) (921, 2564) (931, 2581) (941, 2607) (951, 2628) (961, 2668) (971, 2675) (981, 2700) (991, 2750)
            };
        \end{axis}
    \end{tikzpicture}

    \begin{tikzpicture}
        \begin{axis}[
            title = Одна строка длины 10 вторая - переменной длины,
            legend pos = north west,
            xlabel=len, 
            ylabel=ms,
            grid = major,
            width = 0.8\paperwidth,
            height = 0.38\paperheight,
            line width = 1
        ]
            \legend{
                Дамерау-Левенштейна матричный,
                Дамерау-Левенштейна рекурсивный
            };
            \addplot[blue] coordinates {
                (1, 32) (2, 14) (3, 18) (4, 32) (5, 107) (6, 58) (7, 41) (8, 43) (9, 99)
            };

            \addplot[red] coordinates {
                (1, 19) (2, 65) (3, 474) (4, 2537) (5, 11512) (6, 36962) (7, 107236) (8, 355532) (9, 902894)
            };
        \end{axis}
    \end{tikzpicture}
    \newpage
    \section{Заключение}

    В ходе данной работы было выполнено сравнение трех алгоритмов нахождения минимального редакционного расстояния для двух строк: расстояния Левенштейна матричным методом, расстояния Дамерау-Левенштейна матричным и рекурсивным методами. Было проведено сравнение времени всех алгоритмов для случая, когда две строки имееют одинаковую длину и для случая, когда одна строка небольшая, а вторая варьрируется. В ходе этого исследования были сделаны следующие выводы:

    \begin{enumerate}
        \item[1)] Рекурсивная реализация сильно проигрывает матричной реализации, и ее невыгодно использовать ни в одном случае
        \item[2)] Матричный алгоритм нахождения расстояния Дамерау-Левенштейна работает немного медленнее, чем Левенштейна за счет рассчета дополнительного действия смены соседних букв
    \end{enumerate}
\end{document}
