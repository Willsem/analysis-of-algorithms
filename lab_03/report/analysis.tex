\newpage
\section{Аналитическая часть}

Алгоритмы сортировки имеют большое практическое применение. Их можно встретить почти везде,
где речь идет об обработке и хранении больших объемов информации. Некоторые задачи обработки
данных решаются проще, если данные упорядочены.

\subsection{Описание задачи}

\textbf{Сортировка} - это процесс упорядочения некоторого множества элементов,
на котором определены отношения порядка >, <, >=, <= (по возрастанию или убыванию) \cite{pav}.
При выборе алгоритма сортировки необходимо выбрать тот алгоритм, который будет проделывать
минимум операций над данными и тем самым максимально быстро получать необходимый результат -
отсортированный список.

\textbf{Трудоемкость алгоритма} - это зависимость количества операций от количества
данных, с которыми алгоритм работает. Список действий, цена которых считается за 1:

$$
+, -, *, /, \%, =, ==, !=, <, >, <=, >=, [], +=
$$

На данный момент существует большое количесво алгоритмов сортировки, которые отличаются
своей трудоемкостью \cite{knuth}. Первые прототипы современных методов сортировки появились уже
в XIX веке. К 1890 году для ускорения обработки данных переписи населения в США американец
Герман Холлерит создал первый статистический табулятор — электромеханическую машину,
предназначенную для автоматической обработки информации, записанной на перфокартах
\cite{eniac}.
В дальнейшем история алгоритмов оказалась связана с развитием электронно-вычислительных машин.

За последние 70 лет появилось множество алгоритмов сортировок для компьютера.\cite{knuth}

\subsection{Выводы}

В данной работе стоит задача реализации 3 алгоритмов сортировки: сортировка пузырьком,
шейкером и быстрая сортировка. Необходимо теоретически оценить трудоемкость этих алгоритмов
и проверить все вычисления экспериментально.
