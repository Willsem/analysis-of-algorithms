\newpage
\section{Технологическая часть}

Необходимо выбрать средства для разработки алгоритмов и подготовить тесты.

\subsection{Требования к программному обеспечению}

Программное обеспечение должно обеспечивать замер процессорного времени выполнения каждого
алгоритма. Проводятся замеры для случайно генерируемых массивов размерности до 10000.

\subsection{Средства реализации}

В качестве языка программирования был выбран C++. Данный язык имеет высокую скорость и
богатую стандартную библиотеку, содержащую необходимые контейнеры для данной работы.
Программа, написанная на C++, будет доступна на всех платформах.

Время замерялось с помощью библиотеки chrono, которая измеряет процессорное
время \cite{chrono}.

\subsection{Листинг кода}

Результаты разработки указаны на листингах \ref{list:bubble}, \ref{list:shaker},
\ref{list:qsort}.

\begin{lstlisting}[caption=Сортировка пузырьком,label=list:bubble]
template < typename Type >
void BubbleSort< Type >::sort(std::vector< Type >& array,
                              bool (*comp)(Type, Type))
{
    for (int i = 0; i < array.size(); ++i) {
        for (int j = 0; j < array.size() - i - 1; ++j) {
            if (comp(array[j], array[j + 1])) {
                std::swap(array[j], array[j + 1]);
            }
        }
    }
}
\end{lstlisting}

\begin{lstlisting}[caption=Сортировка шейкером,label=list:shaker]
template < typename Type >
void ShakerSort< Type >::sort(std::vector< Type >& array,
                              bool (*comp)(Type, Type))
{
    int left = 0;
    int right = array.size() - 1;

    while (left <= right) {
        for (int i = left; i < right; ++i) {
            if (comp(array[i], array[i + 1])) {
                std::swap(array[i], array[i + 1]);
            }
        }
        --right;

        for (int i = right; i > left; --i) {
            if (comp(array[i - 1], array[i])) {
                std::swap(array[i - 1], array[i]);
            }
        }
        ++left;
    }
}
\end{lstlisting}

\begin{lstlisting}[caption=Быстрая сортировка,label=list:qsort]
template < typename Type >
void QSort< Type >::recursive(std::vector< Type >& array,
                              int start, int finish,
                              bool (*comp)(Type, Type))
{
    if (start < finish) {
        int left = start, right = finish;
        Type middle = array[(left + right) >> 1];

         while (left <= right) {
            while (comp(middle, array[left])) left++;
            while (comp(array[right], middle)) right--;

            if (left <= right) {
                std::swap(array[left], array[right]);
                ++left;
                --right;
            }
        }

        if (start < right) recursive(array, start, right, comp);
        if (left < finish) recursive(array, left, finish, comp);
    }
}
\end{lstlisting}

\subsection{Тестирование}

Для тестирования программы были заготовлены следующие тесты на таблице \ref{table:test}.

\begin{table}[H]
    \centering
    \caption{Тесты для алгоритмов}
    \label{table:test}
    \begin{tabular}{|c|c|}
        \hline
        Входной массив & Ожидаемый результат \\
        \hline
        1, 2, 3, 4, 5, 6, 7, 8, 9 & 1, 2, 3, 4, 5, 6, 7, 8, 9 \\
        \hline
        9, 8, 7, 6, 5, 4, 3, 2, 1 & 1, 2, 3, 4, 5, 6, 7, 8, 9 \\
        \hline
        4, 2, 7, 4, 8, 2, 4, 8, 1 & 1, 2, 2, 4, 4, 4, 7, 8, 8 \\
        \hline
        5, 5, 5, 5, 5, 5, 5, 5, 5 & 5, 5, 5, 5, 5, 5, 5, 5, 5 \\
        \hline
    \end{tabular}
\end{table}

\subsection{Выводы}

Для сравнения были реализованы 3 алгоритма на выбранном языке
программирования C++. Чтобы проверить правильность работы алгоритмов
были подготовлены тесты.
