\newpage
\section{Технологическая часть}

Необходимо выбрать средства для разработки алгоритмов и подготовить тесты.

\subsection{Требования к программному обеспечению}

Программное обеспечение должно обеспечивать замер процессорного времени выполнения
каждого алгоритма. Проводятся замеры для случайно генерируемых квадратных матриц
размерности до 1000.

\subsection{Средства реализации}

В качестве языка программирования был выбран {\ttfamily C++}.
Данный язык имеет высокую скорость и богатую стандартную библиотеку,
содержащую необходимые контейнеры для данной работы. Программа, написанная на
{\ttfamily C++}, будет доступна на всех платформах.

Время замерялось с помощью библиотеки {\ttfamily chrono}, которая измеряет
процессорное время \cite{chrono}. Для распараллеливания вычислений была
использована библиотека с классом нативных потоков
{\ttfamily std::thread} \cite{thread}.

Тестирование проводится на процессоре с количеством логических потоков равным 4.

\subsection{Листинг кода}

Результаты разработки указаны в листингах \ref{list:vin}, \ref{list:vinth}.

\begin{lstlisting}[caption=Алгоритм Винограда умножения матриц,label=list:vin]
Matrix ModifiedVinogradMultiplication::multiplication(
    const Matrix& first,
    const Matrix& second)
{
    Matrix result;
    if (first[0].size() != second.size()) return result;
    result = CreateMatrix(first.size(), second[0].size(), 0);

    const int m = first.size();
    const int n = second.size();
    const int q = second[0].size();

    Array mulU = CreateArray(m);
    Array mulV = CreateArray(q);

    for (int i = 0; i < m; ++i)
        for (int j = 0; j < n - 1; j += 2)
            mulU[i] -=
                first[i][j] * first[i][j + 1];

    for (int j = 0; j < q; ++j)
        for (int i = 0; i < n - 1; i += 2)
            mulV[j] -=
                second[i][j] * second[i + 1][j];

    for (int i = 0; i < m; ++i) {
        for (int j = 0; j < q; ++j) {
            result[i][j] = mulU[i] + mulV[j];
            for (int k = 0; k < n - 1; k += 2) {
                result[i][j] +=
                    (first[i][k] + second[k + 1][j]) *
                    (first[i][k + 1] + second[k][j]);
            }
        }
    }

    if (n % 2 == 1)
        for (int i = 0; i < m; ++i)
            for (int j = 0; j < q; ++j)
                result[i][j] +=
                    first[i][n - 1] * second[n - 1][j];

    return result;
}
\end{lstlisting}

\begin{lstlisting}[caption=Алгоритм Винограда умножения матриц,label=list:vinth]
void ThreadMultiplication::multiplication(
    const Matrix& first, const Matrix& second,
    Matrix& result,
    const int start, const int finish)
{
    const int m = first.size();
    const int n = second.size();
    const int q = second[0].size();

    Array mulU = CreateArray(m);
    Array mulV = CreateArray(q);

    for (int i = start; i <= finish; ++i)
        for (int j = 0; j < n - 1; j += 2)
            mulU[i] -=
                first[i][j] * first[i][j + 1];

    for (int j = 0; j < q; ++j)
        for (int i = 0; i < n - 1; i += 2)
            mulV[j] -=
                second[i][j] * second[i + 1][j];

    for (int i = start; i <= finish; ++i) {
        for (int j = 0; j < q; ++j) {
            result[i][j] = mulU[i] + mulV[j];
            for (int k = 0; k < n - 1; k += 2) {
                result[i][j] +=
                    (first[i][k] + second[k + 1][j]) *
                    (first[i][k + 1] + second[k][j]);
            }
        }
    }

    if (n % 2 == 1)
        for (int i = start; i <= finish; ++i)
            for (int j = 0; j < q; ++j)
                result[i][j] +=
                    first[i][n - 1] * second[n - 1][j];
}
\end{lstlisting}

\subsection{Тестирование}

Для тестирования программы были заготовлены следующие тесты в таблице
\ref{table:test}.

\begin{table}[H]
    \caption{Тесты для алгоритмов}
    \label{table:test}
    \centering
    \begin{tabular}{|c|c|c|}
        \hline
        Первая матрца & Вторая матрица & Ожидаемый результат \\
        \hline
        1 2 & 1 2 & \ 7 10 \\
        3 4 & 3 4 & 15 22 \\
        \hline
        1 2 3 & 1 2 3 & \ 30\ \ 36\ \ 42 \\
        4 5 6 & 4 5 6 & \ 66\ \ 81\ \ 96 \\
        7 8 9 & 7 8 9 & 102 126 150 \\
        \hline
        1 2 3 & 1 & 14 \\
        4 5 6 & 2 & 32 \\
              & 3 & \\
        \hline
    \end{tabular}
\end{table}

\subsection{Выводы}

Для сравнения были реализованы 3 алгоритма на выбранном языке
программирования {\ttfamily C++}. Чтобы проверить правильность работы алгоритмов
были подготовлены тесты.
