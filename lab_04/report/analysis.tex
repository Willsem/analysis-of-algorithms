\newpage
\section{Аналитическая часть}

Умножение матриц активно применяется в областях физики, математики и программирования.
Рассмотрим как можно решить эту задачу.

\subsection{Описание задачи}

Пусть даны две прямоугольные матрицы $A$ и $B$ размерности $l \times m$ и $m \times n$
соответственно, указанные в формуле \ref{ab}.

\begin{equation}\label{ab}
    A =
    \left[
        \begin{matrix}
            a_{11} & a_{12} & \cdots & a_{1m} \\
            a_{21} & a_{22} & \cdots & a_{2m} \\
            \vdots & \vdots & \ddots & \vdots \\
            a_{l1} & a_{l2} & \cdots & a_{lm} \\
        \end{matrix}
    \right],
    B =
    \left[
        \begin{matrix}
            b_{11} & b_{12} & \cdots & b_{1n} \\
            b_{21} & b_{22} & \cdots & b_{2n} \\
            \vdots & \vdots & \ddots & \vdots \\
            b_{m1} & b_{m2} & \cdots & b_{mn} \\
        \end{matrix}
    \right]
\end{equation}

Тогда матрица $C$ будет размерностью $l \times n$ в формуле \ref{c}, в которой
каждый элемент равен выражению из формулы \ref{res} \cite{litr}.

\begin{equation}\label{c}
    C =
    \left[
        \begin{matrix}
            c_{11} & c_{12} & \cdots & c_{1n} \\
            c_{21} & c_{22} & \cdots & c_{2n} \\
            \vdots & \vdots & \ddots & \vdots \\
            c_{l1} & c_{l2} & \cdots & c_{ln} \\
        \end{matrix}
    \right]
\end{equation}

\begin{equation}\label{res}
    c_{ij} = \sum_{k=1}^m a_{ik} \cdot b_{kj}, i = \overline{1;l}, j = \overline{1;n}
\end{equation}

Операция умножения двух матриц выполнима только в том случае, если число столбцов в
первом сомножителе равно числу строк во втором; в этом случае говорят, что матрицы
согласованы. В частности, умножение всегда выполнимо, если оба сомножителя —
квадратные матрицы одного и того же порядка \cite{litr}.

Таким образом, из существования произведения $A \times B$ вовсе не следует
существование произведения $B \times A$ \cite{litr}.

Помимо обычного перемножения матриц по формуле существуют модификации, работающие
быстрее. Рассмотрим в данной лабораторной работе алгоритм Винограда, являющийся одним
из самых эффективных по времени алгоритмов умножения матриц \cite{haskell},
и ее оптимизацию. Этот алгоритм основывается на подготовке вычислений перед вычислением
результирующей матрицы. Если разложить формулу \ref{res} на суммы, то получается
результат, видимый в формуле \ref{vinres}.

\begin{equation}\label{vinres}
    c_{ij} =
    \sum_{k=1}^{\frac{n}{2}} (a_{i,2k-1} + b_{2k,j}) \cdot (a_{i,2k} + b_{2k-1,j}) -
    \underbrace{\sum_{k=1}^{\frac{n}{2}} a_{i,2k-1} \cdot a_{i,2k} -
    \sum_{k=1}^{\frac{n}{2}} b_{2k-1,j} \cdot b_{2k,j}}_\text{Можно вычислить заранее}
\end{equation}

Таким образом, можно заранее вычислить две последние суммы, поскольку они вычисляются
многократно для каждой строки в одном столбце в случае первой и для каждого столбца
из одной строки в случае второй сумм, что уменьшает долю умножения\cite{haskell}. Также
можно заметить, что вычисление каждого нового элемента результирующей матрицы не
влияет на вычисление следующих, то есть каждый элемент матрицы считается отдельно.
По этой причине можно проделать действия по распараллеливанию вычислений и тем самым
увеличить скорость расчета результата.

\subsection{Выводы}

Умножение матриц необходимый инструмент, для которого есть пути ускорения вычислений
за счет уменьшения доли умножения и распараллеливания вычислений.
