\newpage
\section{Аналитическая часть}

Изучим алгоритмы поиска подстроки в строке.

\subsection{Описание задачи}

Поиск строки формально определяется следующим образом. Пусть задан массив$S$ элементов и массив $X$ элементов. Поиск строки обнаруживает первое вхождение $X$ в $S$, результатом будем считать индекс $i$, указы- вающий на первое с начала строки (с начала массива $S$) совпадение со словом.

\subsection{Пути решения}

\subsubsection{Алгоритм Кнута-Морриса-Пратта}

Алгоритм был разработан Кнутом и Праттом и независимо от них
Моррисом в 1977 г.

После частичного совпадения начальной части подстроки $X$ с
соответствующими символами строки $S$ мы фактически знаем
пройденную часть строки и может «вычислить» некоторые сведения
(на основе самого подстроки $X$), с помощью которых потом быстро
продвинемся по тексту.

Идея КМП-поиска -- при каждом несовпадении двух символов текста и
образа образ сдвигается на самое длинное совпадение начала с концом
префикса (не учитывая тривиальное совпадение самого с собой)

\paragraph{Пример}

Создается массив сдвигов (таблица \ref{table:example1}).

\begin{table}[H]
    \centering
    \caption{Массив сдвигов}
    \label{table:example1}
    \begin{tabular}{|c|c|c|l|l|l|}
    \hline
    0 & 1 & 2 & 3 & 4 & 5 \\ \hline
    a & b & c & a & b & d \\ \hline
    0 & 0 & 0 & 1 & 2 & 0 \\ \hline
    \end{tabular}
\end{table}

В таблице \ref{table:kmp} представлена работа алгоритма.

\begin{table}[H]
    \centering
    \caption{Алгоритм КМП}
    \label{table:kmp}
    \begin{tabular}{|l|l|l|l|l|l|l|l|l|l|l|l|l|l|l|l|}
    \hline
    cтрока & a & b & c & a & b & e  & a & b & c & a & b & c & a & b & d\\
    \hline
    подстрока & a & b & c & a & b & \cellcolor[HTML]{FE0000}d & & & & & & & & & \\
    \hline
    подстрока & & & & a & b & \cellcolor[HTML]{FE0000}c & a & b & d & & & & & & \\
    \hline
    подстрока & & & & & & \cellcolor[HTML]{FE0000}a & b & c & a & b & d & & & & \\
    \hline
    подстрока & & & & & & & a & b & c & a & b & \cellcolor[HTML]{FE0000}d & & & \\
    \hline
    подстрока & & & & & & & & & & \cellcolor[HTML]{34FF34}a & \cellcolor[HTML]{34FF34}b & \cellcolor[HTML]{34FF34}c & \cellcolor[HTML]{34FF34}a & \cellcolor[HTML]{34FF34}b & \cellcolor[HTML]{34FF34}d \\
    \hline
    \end{tabular}
\end{table}

\subsubsection{Алгоритм Бойера-Мура}

Алгоритм поиска строки Бойера -- Мура считается наиболее быстрым
среди алгоритмов общего назначения, предназначенных для поиска
подстроки в строке. Был разработан Бойером и Муром в 1977 году.
Преимущество этого алгоритма в том, что ценой некоторого количества
предварительных вычислений над шаблоном (но не над строкой, в которой
ведётся поиск) шаблон сравнивается с исходным текстом не во всех
позициях -- часть проверок пропускаются как заведомо не дающие
результата.

Основная идея алгоритма -- начать поиск не с начала, а с конца
подстроки. Наткнувшись на несовпадение, мы просто смещаем подстроку
до самого правого вхождения данного символа, не учитывая последний.

\paragraph{Пример}

Создается массив прыжков (таблица \ref{table:example2}).

\begin{table}[H]
    \centering
    \caption{Массив прыжков}
    \label{table:example2}
    \begin{tabular}{|c|c|c|l|l|l|}
    \hline
    0 & 1 & 2 & 3 & 4 & 5 \\ \hline
    a & b & c & a & b & d \\ \hline
    0 & 0 & 0 & 1 & 2 & 0 \\ \hline
    \end{tabular}
\end{table}

В таблице \ref{table:bm} представлена работа алгоритма.

\begin{table}[H]
    \centering
    \caption{Алгоритм БМ}
    \label{table:bm}
    \begin{tabular}{|l|l|l|l|l|l|l|l|l|l|l|l|l|l|l|l|}
    \hline
    cтрока & a & b & c & a & b & e & a & b & c & a & b & c & a & b & d \\
    \hline
    подстрока & a & b & c & a & b & \cellcolor[HTML]{FE0000}d & & & & & & & & & \\
    \hline
    подстрока & & & & & & & a & b & c & a & b & \cellcolor[HTML]{FE0000}d & & & \\
    \hline
    подстрока & & & & & & & & & & \cellcolor[HTML]{34FF34}a & \cellcolor[HTML]{34FF34}b & \cellcolor[HTML]{34FF34}c & \cellcolor[HTML]{34FF34}a & \cellcolor[HTML]{34FF34}b & \cellcolor[HTML]{34FF34}d \\
    \hline
    \end{tabular}
\end{table}

\subsection{Выводы}

Изучены алгоритмы поиска подстроки в строке, необходимо реализовать их.
