\documentclass[a4paper,12pt]{article}
\usepackage[T2A]{fontenc}
\usepackage[utf8]{inputenc}
\usepackage[english,russian]{babel}
\usepackage{listings}

\usepackage[table]{xcolor}

\usepackage{amsmath}
\usepackage{MnSymbol}
\usepackage{wasysym}
\usepackage{indentfirst}
\usepackage[unicode, pdftex]{hyperref}

\usepackage{pgfplots}
\pgfplotsset{compat=1.9}

\usepackage{geometry}
\geometry{left=2cm}
\geometry{right=1.5cm}
\geometry{top=1cm}
\geometry{bottom=2cm}

\usepackage{graphicx}
\graphicspath{{img/}}
\DeclareGraphicsExtensions{.pdf,.png,.jpg}

\usepackage{float}

\newcommand{\anonsection}[1]{\section*{#1}\addcontentsline{toc}{section}{#1}}

% переименовываем  список литературы в "список используемой литературы"
\addto\captionsrussian{\def\refname{Список используемой литературы}}

\lstset{
    language=C++,
    numbers=left,
    frame=single,
    texcl=true,
    basicstyle=\ttfamily
}
