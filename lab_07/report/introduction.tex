\newpage
\anonsection{Введение}

Те, кому приходиться часто работать с текстовыми редакторами,
часто пользуются функцией нахождения слов в тексте,
которая существенно облегчает редактирование документов и
поиск нужной информации. Все современные текстовые редакторы
поддерживают функционал поиска и замены
текстовых фрагментов \cite{office}.

Функции поиска входят во многие языки
программирования высокого уровня – чтобы найти строчку
в небольшом тексте используется встроенная функция \cite{cpp}.

Цель данной работы изучить и разработать алгоритм поиска
подстроки в строке.

Задачи данной работы:

\begin{itemize}
    \item изучить основные алгоритмы, решающих задачу поиска;
    \item реализовать данные алгоритмы.
\end{itemize}
