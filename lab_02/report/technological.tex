\newpage
\section{Технологическая часть}

Необходимо выбрать средства для разработки алгоритмов и подготовить тесты.

\subsection{Требования к программному обеспечению}

Программное обеспечение должно обеспечивать замер процессорного времени выполне- ния каждого алгоритма. Проводятся замеры для случайно генерируемых квадратных матриц
размерности до 1000.

\subsection{Средства реализации}

В качестве языка программирования был выбран C++.
Данный язык имеет высокую скорость и богатую стандартную библиотеку,
содержащую необходимые контейнеры для данной работы. Программа, написанная на C++,
будет доступна на всех платформах.

Время замерялось с помощью библиотеки chrono, которая измеряет процессорное время \cite{chrono}.

\subsection{Листинг кода}

Результаты разработки указаны на листингах \ref{list:default}, \ref{list:vin},
\ref{list:modvin}.

\begin{lstlisting}[caption=Стандартный алгоритм умножения матриц,label=list:default]
Matrix DefaultMultiplication::multiplication(
    const Matrix& first,
    const Matrix& second)
{
    Matrix result;
    if (first[0].size() != second.size()) return result;
    result = CreateMatrix(first.size(), second[0].size(), 0);

    for (int i = 0; i < first.size(); ++i) {
        for (int j = 0; j < second[0].size(); ++j) {
            for (int k = 0; k < second.size(); ++k) {
                result[i][j] = result[i][j] +
                    first[i][k] * second[k][j];
            }
        }
    }

    return result;
}
\end{lstlisting}

\begin{lstlisting}[caption=Алгоритм Винограда умножения матриц,label=list:vin]
Matrix VinogradMultiplication::multiplication(
    const Matrix& first,
    const Matrix& second)
{
    Matrix result;
    if (first[0].size() != second.size()) return result;
    result = CreateMatrix(first.size(), second[0].size(), 0);

    const int m = first.size();
    const int n = second.size();
    const int q = second[0].size();

    Array mulU = CreateArray(m);
    Array mulV = CreateArray(q);

    for (int i = 0; i < m; ++i)
        for (int j = 0; j < n >> 1; ++j)
            mulU[i] = mulU[i] +
                first[i][2 * j] * first[i][2 * j + 1];

    for (int j = 0; j < q; ++j)
        for (int i = 0; i < n >> 1; ++i)
            mulV[j] = mulV[j] +
                second[2 * i][j] * second[2 * i + 1][j];

    for (int i = 0; i < m; ++i) {
       for (int j = 0; j < q; ++j) {
            result[i][j] = -mulU[i] - mulV[j];
            for (int k = 0; k < n >> 1; ++k) {
                result[i][j] = result[i][j] +
                    (first[i][2 * k] + second[2 * k + 1][j]) *
                    (first[i][2 * k + 1] + second[2 * k][j]);
            }
        }
    }

    if (n % 2 == 1)
        for (int i = 0; i < m; ++i)
            for (int j = 0; j < q; ++j)
                result[i][j] = result[i][j] +
                    first[i][n - 1] * second[n - 1][j];

    return result;
}
\end{lstlisting}

\begin{lstlisting}[caption=Оптимизированный алгоритм Винограда умножения матриц,label=list:modvin]
Matrix ModifiedVinogradMultiplication::multiplication(
    const Matrix& first,
    const Matrix& second)
{
    Matrix result;
    if (first[0].size() != second.size()) return result;
    result = CreateMatrix(first.size(), second[0].size(), 0);

    const int m = first.size();
    const int n = second.size();
    const int q = second[0].size();

    Array mulU = CreateArray(m);
    Array mulV = CreateArray(q);

    for (int i = 0; i < m; ++i)
        for (int j = 0; j < n - 1; j += 2)
            mulU[i] -=
                first[i][j] * first[i][j + 1];

    for (int j = 0; j < q; ++j)
        for (int i = 0; i < n - 1; i += 2)
            mulV[j] -=
                second[i][j] * second[i + 1][j];

    for (int i  0; i < m; ++i) {
        for (int j = 0; j < q; ++j) {
            result[i][j] = mulU[i] + mulV[j];
            for (int k = 0; k < n - 1; k += 2) {
                result[i][j] +=
                    (first[i][k] + second[k + 1][j]) *
                    (first[i][k + 1] + second[k][j]);
            }
        }
    }

    if (n % 2 == 1)
        for (int i = 0; i < m; ++i)
            for (int j = 0; j < q; ++j)
                result[i][j] +=
                    first[i][n - 1] * second[n - 1][j];

    return result;
}
\end{lstlisting}

\subsection{Тестирование}

Для тестирования программы были заготовлены следующие тесты на таблице
\ref{table:test}.

\begin{table}[H]
    \caption{Тесты для алгоритмов}
    \label{table:test}
    \centering
    \begin{tabular}{|c|c|c|}
        \hline
        Первая матрца & Вторая матрица & Ожидаемый результат \\
        \hline
        1 2 & 1 2 & \ 7 10 \\
        3 4 & 3 4 & 15 22 \\
        \hline
        1 2 3 & 1 2 3 & \ 30\ \ 36\ \ 42 \\
        4 5 6 & 4 5 6 & \ 66\ \ 81\ \ 96 \\
        7 8 9 & 7 8 9 & 102 126 150 \\
        \hline
        1 2 3 & 1 & 14 \\
        4 5 6 & 2 & 32 \\
              & 3 & \\
        \hline
    \end{tabular}
\end{table}

\subsection{Выводы}

Для сравнения были реализованы 3 алгоритма на выбранном языке
программирования C++. Чтобы проверить правильность работы алгоритмов
были подготовлены тесты.
