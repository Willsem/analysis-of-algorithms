\newpage
\section{Технологическая часть}

Реализуем необхоимые алгоритмы.

\subsection{Требования к программному обеспечению}

Программное обеспечение должно предоставлять возможность замера процессорного времени выполнения алгоритмов.
Требуется строить графы для различных данных.

\subsection{Средства реализации}

В качестве языка программирования был выбран {\ttfamily C++}.
Данный язык имеет высокую скорость и богатую стандартную библиотеку,
содержащую необходимые контейнеры для данной работы. Программа, написанная на
{\ttfamily C++}, будет доступна на всех платформах.

\subsection{Листинг кода}

На листингах \ref{lst:fullcall}, \ref{lst:full} и \ref{lst:dist} представлен алгоритм полного рекурсивного
перебора всех путей.

\begin{lstlisting}[caption=Вызов алгоритма рекурсивного перебора,label=lst:fullcall]
std::pair< std::vector< size_t >, size_t > FullSearch::way(
    const Matrix< size_t >& distances)
{
    minDist = -1;

    std::vector< size_t > indexes;
    for (int i = 0; i < distances.size(); ++i)
        indexes.push_back(i);

    recursive(distances, indexes, 1);

    return std::pair< std::vector< size_t >, size_t >(minWay,
                                                      minDist);
}
\end{lstlisting}

\begin{lstlisting}[caption=Алгоритм рекурсивного перебора,label=lst:full]
void FullSearch::recursive(const Matrix< size_t >& distances,
                           std::vector< size_t >& indexes,
                           size_t curIndex)
{
    if (curIndex >= indexes.size()) {
        size_t dist = distance(distances, indexes);

        if (dist < minDist) {
            minDist = dist;
            minWay = indexes;
        }

        return;
    }

    recursive(distances, indexes, curIndex + 1);

    for (int i = curIndex + 1; i < indexes.size(); ++i) {
        std::swap(indexes[curIndex], indexes[i]);
        recursive(distances, indexes, curIndex + 1);
        std::swap(indexes[curIndex], indexes[i]);
    }
}
\end{lstlisting}

\begin{lstlisting}[caption=Подсчет пути,label=lst:dist]
size_t FullSearch::distance(const Matrix< size_t >& distances,
                            const std::vector< size_t >& indexes)
{
    size_t dist = 0;
    for (int i = 0; i < indexes.size() - 1; ++i) {
        dist += distances[indexes[i]][indexes[i + 1]];
    }
    dist += distances[indexes[0]][indexes[indexes.size() - 1]];
    return dist;
}
\end{lstlisting}

На листинге \ref{lst:ant} представлен муравьиный алгоритм.

\begin{lstlisting}[caption=Муравьиный алгоритм,label=lst:ant]
std::pair< std::vector< size_t >, size_t > AntSearch::way(
    const Matrix< size_t >& distances,
    double alpha,
    double beta,
    double rho,
    double tmax)
{
    srand(time(NULL));

    Matrix< double > pheromone(distances.size(), 0.1);

    size_t minDist = -1;
    std::vector< size_t > minWay(distances.size());

    for (int time = 0; time < tmax; ++time) {
        std::vector< int > ants(distances.size());
        std::vector< std::vector< double > > delta;

        for (int i = 0; i < distances.size(); ++i) {
            ants[i] = i + 1;
            delta.push_back(
                std::vector< double >(distances.size())
            );
        }

        for (int ant = 0; ant < ants.size(); ++ant) {
            int count = 0;
            std::vector< int > cities(distances.size());
            std::vector< size_t > way(distances.size());

            cities[ant] = 1;
            way[0] = ant;

            int len = 0;

            while (count + 1 < distances.size()) {
                std::vector< double > p(distances.size());

                for (int j = 0; j < distances.size(); ++j) {
                    if (cities[j] == 0) {
                        p[j] =
                            std::pow(
                                pheromone[way[count]][j],
                                alpha
                            ) *
                            std::pow(
                                distances[way[count]][j],
                                beta
                            );

                        double all = 0;

                        for (int q = 0; q < count; ++q) {
                            all +=
                                std::pow(
                                    pheromone[way[q]][j],
                                    alpha
                                ) *
                                std::pow(
                                    distances[way[q]][j],
                                    beta
                                );
                        }

                        p[j] /= all;
                    } else {
                        p[j] = 0;
                    }
                }

                std::vector< int > arr(
                    distances.size() - count - 1
                );
                int cyc = 0;

                for (int i = 0; i < distances.size(); ++i) {
                    if (cities[i] == 0) {
                        arr[cyc] = i;
                        ++cyc;
                    }
                }

                int rdm = rand() % (distances.size() - count - 1);
                len += distances[way[count]][arr[rdm]];
                ++count;
                way[count] = arr[rdm];
                cities[arr[rdm]] = 1;
            }

            len += distances[way[0]][way[distances.size() - 1]];

            for (int i = 0; i < distances.size() - 1; ++i) {
                delta[way[i]][way[i + 1]] += minDist / len;
                delta[way[i + 1]][way[i]] += minDist / len;
            }

            if (len < minDist) {
                minDist = len;
                minWay = way;
            }
        }

        for (int i = 0; i < distances.size() - 1; ++i) {
            for (int j = i + 1; j < distances.size(); ++j) {
                pheromone[i][j] *= (1 - rho);
                pheromone[i][j] += delta[i][j];
            }
        }
    }

    return std::pair< std::vector< size_t >, size_t >(minWay,
                                                      minDist);
}
\end{lstlisting}

\subsection{Тестирование}

Для тестирования были заготовлены следующие тесты, которые представлены
на таблице \ref{table:test}.

\begin{table}[H]
    \caption{Тесты}
    \label{table:test}
    \centering
    \begin{tabular}{|l||c|}
        \hline
        Входной граф & Ожидаемый результат \\
        \hline
        \hline
        \hline
        4 & \\
        0 1 2 3 & \\
        1 0 3 4 & 0 1 2 3 \\
        2 3 0 5 & dist = 12 \\
        3 4 5 0 & \\
        \hline
        5 & \\
        0 1 -1 3 4 & \\
        1 0 3 4 5 & 0 1 3 4 2 \\
        -1 3 0 5 -1 & dist = 10 \\
        3 4 5 0 7 & \\
        4 5 -1 7 0 & \\
        \hline
        5 & \\
        0 5 8 9 2 & \\
        5 0 3 3 4 & 0 2 1 3 4 \\
        8 3 0 7 6 & dist = 17 \\
        9 3 7 0 1 & \\
        2 4 6 1 0 & \\
        \hline
    \end{tabular}
\end{table}

\subsection{Выводы}

Разаботаны алгоритмы полного перебора и муравьиный алгоритм, необходимо исследовать муравьиный алгоритм
с помощью алгоритма перебора, предварительно его протестировав.
