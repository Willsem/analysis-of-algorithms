\newpage
\section{Аналитическая часть}

Рассмотрим задачу коммивояжера и пути ее решений.

\subsection{Описание задачи}

Задача коммивояжёра –- одна из самых известных задач дискретной оптимизации. Задача заключается в нахождении
самого выгодного маршрута, проходящего через указанные города хотя бы по одному разу с последующим возвратом
в исходный город \cite{Levitin}.

Постановка задачи коммивояжёра: пусть дана сеть из $N$ городов. Коммивояжёр, выходящий из какого-нибудь города,
желает посетить $N-1$ других городов и вернуться в изначальный пункт. Расстояния между всеми этими городами
известны. Требуется установить, в каком порядке коммивояжёру следует посетить города, чтобы суммарное пройденное
расстояние было минимальным.

\subsection{Пути решения}

Для решения задачи коммивояжера разработаны различные алгоритмы:

\begin{enumerate}
    \item метод полного перебора;
    \item жадный алгоритм;
    \item муравьиный алгоритм.
\end{enumerate}

В данной работе проводится исследование двух из них: метода полного перебора и муравьиного алгоритма.

Алгоритм полного перебора осуществляет поиск всех решений путём перебора всех вариантов в количестве $N!$
путей, позволяя получить глобальный минимум по всему графу. Главный недостаток метода полного перебора –- временные затраты.

Муравьиный алгоритм (алгоритм оптимизации подражанием муравьиной колонии) представляет собой имитацию поведения колонии
муравьёв в природе. В основе муравьиного алгоритма лежит вероятностный подход к поиску оптимального пути, однако имеют
большое значение дополнительные критерии.

Преимуществами алгоритма являются невысокая погрешность найденного решения, низкие временные затраты при работе
с графами большой размерности, модифицируемость алгоритма и возможность распараллеливания \cite{ant}.

У муравья есть 3 чувства:

\begin{enumerate}
    \item обоняние -– муравей может чуять феромон и его концентрацию на ребре;
    \item зрение –- муравей может оценить длину ребра;
    \item память –- муравей запоминает посещенные города.
\end{enumerate}

При старте матрица феромонов $\tau$ инициализируется равномерно некоторой константой.
Если муравей $k$ находится в городе $i$ и выбирает куда пойти, то делает это по вероятностному правилу, указанному
на формуле \ref{eq:ant}.

\begin{equation}\label{eq:ant}
    P_{x,y}(t) =
    \begin{cases}
        \frac{\tau_{ij}(t)^\alpha \cdot \nu_{ij}^\beta}{\sum_q^\text{cities} \tau{iq}(t)^\alpha \nu_{iq}^\beta},
        \text{ если город } j \text{ в списке целей} \\
        0, \text{ иначе} \\
    \end{cases},
\end{equation}

где cities -- список посещенных городов, $\alpha, \beta$ -- весовые коэффициенты,
важность феромона и привлекательность ребра соотвественно.

Ночью пересчитываются феромоны по формуле \ref{eq:fer}.

\begin{equation}\label{eq:fer}
    \tau(t+1) = \tau_{ij}(t) \cdot (1 - \rho) + \Delta \tau_{ij}(t)
\end{equation}

\begin{equation}
    \Delta \tau_{ij}(t) = \sum_{k=1}^n \Delta \tau_{k,ij}(t)
\end{equation}

\begin{equation}
    \Delta \tau_{k,ij}(t) =
    \begin{cases}
        \frac{Q}{L_k}, \text{ если ребро } ij \text{ в маршруте } k \text{-го муравья} \\
        0, \text{ иначе} \\
    \end{cases}
\end{equation}

где $L_k$ -- длина маршрута $k$-го муравья, $\rho$ -- коэффициент испарения феромона, $Q$ - нормированная константа
порядка длины наилучшего маршрута

\subsection{Выводы}

Изучив алгоритм муравья, необходимо разработать его и провести исследование.
