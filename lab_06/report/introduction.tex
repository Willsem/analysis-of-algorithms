\newpage
\anonsection{Введение}

Задача коммивояжера занимает особое место в комбинаторной оптимизации и исследовании операций.
Исторически она была одной из тех задач, которые послужили толчком для развития этих направлений.
Простота формулировки, конечность множества допустимых решений, но колоссальные затраты на полный
перебор до сих пор подталкивают к разработке все новых и новых численных методов.

С точки зрения практического применения, она не представляет интерес.
Куда важнее дополнения задачи для транспорта и логистики, когда несколько
транспортных средств ограниченной грузоподъемности должны обслуживать клиентов,
посещая их в заданные временные окна \cite{kommivoyadjer}.

Целью данной работы является создание приложения для наглядного представления работы муравьиного
алгоритма и для проведения вычислительных экспериментов.

В данной работе ставятся следующие задачи:

\begin{itemize}
    \item изучение существующих алгоритмов задачи коммивояжера;
    \item реализация муравьиного алгоритма;
    \item проведение вычислительного эксперимента.
\end{itemize}
