\newpage
\section{Технологическая часть}

Необходимо изучить и реализовать онвеерную разработку для умножения матриц.

\subsection{Требования к программному обеспечению}

Программное обеспечение должно обеспечивать замер процессорного времени выполнения
каждого алгоритма. Проводятся замеры для случайно генерируемых квадратных матриц
размерности до 1000.

\subsection{Средства реализации}

В качестве языка программирования был выбран {\ttfamily C++}.
Данный язык имеет высокую скорость и богатую стандартную библиотеку,
содержащую необходимые контейнеры для данной работы. Программа, написанная на
{\ttfamily C++}, будет доступна на всех платформах.

Время замерялось с помощью библиотеки {\ttfamily chrono}, которая измеряет
процессорное время \cite{chrono}. Для распараллеливания вычислений была
использована библиотека с классом нативных потоков
{\ttfamily std::thread} \cite{thread}.

Тестирование проводится на 2,7 GHz 2‑ядерном процессоре Intel Core i5 с количеством логических потоков равным 4.

\subsection{Листинг кода}

На листингах \ref{lst:inputMatrix1}, \ref{lst:inputMatrix2} предствавлен код для
выделения памяти и ввода двух матриц.

\begin{lstlisting}[caption=Ввод первой матрицы,label=lst:inputMatrix1]
void Multiplication::inputMatrix1(std::ifstream& input)
{
    input >> n1 >> m1;

    for (int i = 0; i < n1; ++i) {
        Array line;

        for (int j = 0; j < m1; ++j) {
            int element;
            input >> element;
            line.push_back(element);
        }

        first.push_back(line);
    }
}
\end{lstlisting}

\begin{lstlisting}[caption=Ввод второй матрицы,label=lst:inputMatrix2]
void Multiplication::inputMatrix2(std::ifstream& input)
{
    input >> n2 >> m2;

    for (int i = 0; i < n2; ++i) {
        Array line;

        for (int j = 0; j < m2; ++j) {
            int element;
            input >> element;
            line.push_back(element);
        }

        second.push_back(line);
    }
}
\end{lstlisting}

На листинге \ref{lst:createResult} представлен код для выделения памяти под результат
и массивы { \ttfamily mulU } и { \ttfamily mulV }.

\begin{lstlisting}[caption=Выделение памяти для результата и дополнительных массивов,label=lst:createResult]
void Multiplication::createResult()
{
    if (m1 != n2) return;

    for (int i = 0; i < n1; ++i) {
        Array line;

        for (int j = 0; j < m2; ++j) {
            line.push_back(0);
        }

        result.push_back(line);
    }

    for (int i = 0; i < n1; ++i) {
        mulU.push_back(0);
    }

    for (int i = 0; i < m2; ++i) {
        mulV.push_back(0);
    }
}
\end{lstlisting}

На листингах \ref{lst:mulU}, \ref{lst:mulV} представлен код для вычисления
массивов { \ttfamily mulU } и { \ttfamily mulV }.

\begin{lstlisting}[caption=Подсчет { \ttfamily mulU },label=lst:mulU]
void Multiplication::calcMulU()
{
    for (int i = 0; i < n1; ++i)
        for (int j = 0; j < n2 - 1; j += 2)
            mulU[i] -=
                first[i][j] * first[i][j + 1];
}
\end{lstlisting}

\begin{lstlisting}[caption=Подсчет { \ttfamily mulV },label=lst:mulV]
void Multiplication::calcMulV()
{
    for (int j = 0; j < m2; ++j)
        for (int i = 0; i < n2 - 1; i += 2)
            mulV[j] -=
                second[i][j] * second[i + 1][j];
}
\end{lstlisting}

На листингах \ref{lst:calculate1}, \ref{lst:calculate2} и \ref{lst:check} вычисляется
результат.

\begin{lstlisting}[caption=Подсчет первой половины результата,label=lst:calculate1]
void Multiplication::calculate1()
{
    for (int i = 0; i < (n1 >> 1) + 1; ++i) {
        for (int j = 0; j < m2; ++j) {
            result[i][j] = mulU[i] + mulV[j];
            for (int k = 0; k < n2 - 1; k += 2) {
                result[i][j] +=
                    (first[i][k] + second[k + 1][j]) *
                    (first[i][k + 1] + second[k][j]);
            }
        }
    }
}
\end{lstlisting}

\begin{lstlisting}[caption=Подсчет второй половины результата,label=lst:calculate2]
void Multiplication::calculate2()
{
    for (int i = (n1 >> 1) + 1; i < n1; ++i) {
        for (int j = 0; j < m2; ++j) {
            result[i][j] = mulU[i] + mulV[j];
            for (int k = 0; k < n2 - 1; k += 2) {
                result[i][j] +=
                    (first[i][k] + second[k + 1][j]) *
                    (first[i][k + 1] + second[k][j]);
            }
        }
    }
}
\end{lstlisting}

\begin{lstlisting}[caption=Проверка на нечетность и вычисление если необходимо,label=lst:check]
void Multiplication::check()
{
    if (n2 % 2 == 1)
        for (int i = 0; i < n1; ++i)
            for (int j = 0; j < m2; ++j)
                result[i][j] +=
                    first[i][n2 - 1] * second[n2 - 1][j];
}
\end{lstlisting}

На листингах \ref{lst:pipeline1}, \ref{lst:pipeline2}, \ref{lst:pipeline3} и
\ref{lst:pipeline4} проводятся конвеерные вычисления умножения матриц.

\begin{lstlisting}[caption=Первый конвейер,label=lst:pipeline1]
void Conveyor::functionPipeline1()
{
    while (true)
    {
        std::string buf;
        input >> buf;

        if (input.eof()) {
            stopPipline1 = true;
            return;
        }

        Multiplication mult;
        mult.inputMatrix1(input);
        mult.inputMatrix2(input);

        queuePipeline2.push(mult);
    }
}
\end{lstlisting}

\begin{lstlisting}[caption=Второй конвейер,label=lst:pipeline2]
void Conveyor::functionPipeline2()
{
    while (true)
    {
        if (stopPipline1 &&
            queuePipeline2.empty()) {
            stopPipline2 = true;
            return;
        }

        if (queuePipeline2.empty()) continue;

        Multiplication mult = queuePipeline2.front();
        queuePipeline2.pop();

        mult.createResult();
        mult.calcMulU();
        mult.calcMulV();

        queuePipeline3.push(mult);
    }
}
\end{lstlisting}

\begin{lstlisting}[caption=Третий конвейер,label=lst:pipeline3]
void Conveyor::functionPipeline3()
{
    while (true)
    {
        if (stopPipline2 &&
            queuePipeline3.empty()) {
            stopPipline3 = true;
            return;
        }

        if (queuePipeline3.empty()) continue;

        Multiplication mult = queuePipeline3.front();
        queuePipeline3.pop();

        mult.calculate1();

        queuePipeline4.push(mult);
    }
}
\end{lstlisting}

\begin{lstlisting}[caption=Четвертый конвейер,label=lst:pipeline4]
void Conveyor::functionPipeline4()
{
    while (true)
    {
        if (stopPipline3 &&
            queuePipeline4.empty()) {
            return;
        }

        if (queuePipeline4.empty()) continue;

        Multiplication mult = queuePipeline4.front();
        queuePipeline4.pop();

        mult.calculate2();
        mult.check();

        queueResult.push(mult);
    }
}
\end{lstlisting}

\subsection{Тестирование}

Для тестирования программы были заготовлены следующие тесты в таблице
\ref{table:test}.

\begin{table}[H]
    \caption{Тесты для алгоритмов}
    \label{table:test}
    \centering
    \begin{tabular}{|c|c|c|}
        \hline
        Первая матрца & Вторая матрица & Ожидаемый результат \\
        \hline
        1 2 & 1 2 & \ 7 10 \\
        3 4 & 3 4 & 15 22 \\
        \hline
        1 2 3 & 1 2 3 & \ 30\ \ 36\ \ 42 \\
        4 5 6 & 4 5 6 & \ 66\ \ 81\ \ 96 \\
        7 8 9 & 7 8 9 & 102 126 150 \\
        \hline
        1 2 3 & 1 & 14 \\
        4 5 6 & 2 & 32 \\
              & 3 & \\
        \hline
    \end{tabular}
\end{table}

\subsection{Выводы}

Для исследования конвеерных вычислений был реализован алгоритм на выбранном языке
{ \ttfamily C++ }. Чтобы проверить правильность работы были подготовлены тесты.
