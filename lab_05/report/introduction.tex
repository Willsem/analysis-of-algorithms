\newpage
\anonsection{Введение}

Имеется большое количество важнейших задач, решение которых требует использования огромных вычислительных мощностей,
зачастую недоступных для современных вычислительных систем.

Постоянно появляются новые задачи подобного рода и возрастают требования к точности и к скорости решения прежних задач;
поэтому вопросы разработки и использования сверхмощных компьютеров (называемых суперкомпьютерами) актуальны сейчас и в
будущем \cite{Voevodin}. Но пока эти трудности пока что не удается преодолеть. Из-за этого приходится и эти по пути
создания параллельных вычислительных систем, т.е. систем, в которых предусмотрена одновременная реализация ряда
вычислительных процессов, связанных с решением одной задачи. \cite{Korneev} На современном этапе развития вычислительной
техники такой способ, по-видимому, является одним из основных способов ускорения вычислений.

Многие явления природы характеризуются параллелизмом (одновременным исполнением процессов с применением различных путей и
способов). В частности, в живой природе параллелизм распространен очень широко в дублирующих системах для получения
надежного результата. Параллельные процессы пронизывают общественные отношения, ими характеризуются развитие науки,
культуры и экономики в человеческом обществе. Среди этих процессов особую роль играют параллельные информационные потоки \cite{Conveer}.

Среди параллельных систем различают конвейерные, векторные, матричные, систолические, спецпроцессоры и т.п. В данной
работе используются конвейерные \cite{Korneev}.

В данной работе стоит задача реализации алгоритма Винограда для умножения матриц, сравнение последовательной и
конвейерное реализаций.
